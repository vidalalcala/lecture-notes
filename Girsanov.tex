\documentclass{article}
\usepackage{graphicx}
\usepackage{amsmath}
\usepackage{amsfonts}
\usepackage{amsthm}
\usepackage{enumerate}
\usepackage{fancyhdr}
\usepackage{amstext}
\usepackage{graphicx}


\usepackage{hyperref}
\hypersetup{
    bookmarksopen=false,
    bookmarksnumbered=true,
    bookmarksopenlevel=2
}
\usepackage[greek,english]{babel}
%-----------------------------------------------------------------------------------------------------------------------------

\newtheorem{thm}{Theorem}[section]
\newtheorem{exe}{Exercise}
\newtheorem{cor}[thm]{Corolario}
\newtheorem{lem}[thm]{Lema}
\newtheorem{prop}[thm]{Proposici�n}
\theoremstyle{definition}
\newtheorem{defn}[thm]{Definici�n}
\theoremstyle{remark}
\newtheorem{rem}[thm]{Observaci�n}
\theoremstyle{definition}
\newtheorem{ejm}[thm]{Ejemplo}
\numberwithin{equation}{section}


% MATH -----------------------------------------------------------
\newcommand{\norm}[1]{\left\Vert#1\right\Vert}
\newcommand{\abs}[1]{\left\vert#1\right\vert}
\newcommand{\set}[1]{\left\{#1\right\}}
\newcommand{\Real}{\mathbb{R}}
\newcommand{\C}{\mathbb{C}}
\newcommand{\Int}{\mathbb{Z}}
\newcommand{\Field}{\mathbb {F}}
\newcommand{\Mat}{\mathrm{Mat}}
\newcommand{\Img}{\mathrm{Im}\:}
\newcommand{\Var}{\mathrm{Var}\:}
\newcommand{\diam}{\mathrm{diam}\:}
\newcommand{\inte}{\mathrm{int}\:}
\newcommand{\dist}{\mathrm{dist}\:}
\newcommand{\Cov}{\mathrm{Cov}\:}
\newcommand{\Rea}{\mathrm{Re}\:}
\newcommand{\tr}{\mathrm{tr}\:}
\newcommand{\Alg}{\mathcal{A}}
\newcommand{\F}{\mathcal{F}}
\newcommand{\Mo}{M}
\newcommand{\Mod}{N}
\newcommand{\Id}{\mathcal {I}}
\newcommand{\e}{\mathrm{e}}
\newcommand{\Max}{\mathcal {M}}
\newcommand{\Nat}{\mathbb {N}}
\newcommand{\Rat}{\mathbb {Q}}
\newcommand{\sub}{\subset}
\newcommand{\GL}[1]{\mathrm{GL}(#1)}
\newcommand{\SL}[1]{\mathrm{SL}(#1)}
\newcommand{\End}[1]{\mathrm{End}(#1)}
\newcommand{\Der}[1]{\mathrm{Der}(#1)}
\newcommand{\Hom}[2]{\mathrm{Hom}({#1},{#2})}
\newcommand{\HomM}[2]{\mathrm{Hom}_\Alg({#1},{#2})}
\newcommand{\Homl}[2]{\mathbf{Hom}({#1},{#2})}
\newcommand{\HomMl}[2]{\mathbf{Hom}_\Alg({#1},{#2})}
\newcommand{\Ber}{\mathrm{Ber}}
\newcommand{\Lie}{\mathrm{Lie}}
\newcommand{\eps}{\varepsilon}
\newcommand{\To}{\longrightarrow}
\newcommand{\BX}{\mathbf{B}(X)}
\newcommand{\A}{\mathcal{A}}
\newcommand{\M}{M}
\newcommand{\SM}{\mathcal{M}}
\newcommand{\N}{N}
\newcommand{\SN}{\mathcal{N}}
\newcommand{\Gav}{\mathcal{O}}
\newcommand{\Gavi}{\mathcal{P}}
\newcommand{\J}{\mathcal{J}}
\newcommand{\G}{G}
\newcommand{\SMan}{\mathbf{SMan}}
\newcommand{\Set}{\mathbf{Set}}
\newcommand{\Grp}{\mathbf{Grp}}
\newcommand{\g}{\gamma}
\newcommand{\p}{\lambda}
\newcommand{\Li}{\mathcal{L}}
\newcommand{\Com}[2]{\biggl(
\begin{matrix}
#1\\
#2
\end{matrix}
\biggr)}

\newcommand{\Fun}{\mathbf{F}}
\newcommand{\1}{1}
\newcommand{\gl}{\frak{gl}}

%-----------------------------------------------------------------
\numberwithin{thm}{section} \numberwithin{equation}{section}
% ----------------------------------------------------------------

% ----------------------------------------------------------------
\begin{document}
\title{\textbf{Girsanov Theorem}}
\author{Jose Alcala}
\date{April 2010}
% ----------------------------------------------------------------
\maketitle{}

\section{Derivation}
Assume $P$ is the measure on paths given by the solution of the SDE 
$$
dX=b\:dt + \sigma \: dW\:,
$$
and $Q$ the measure on paths given by the solution of the SDE $dX=\sigma \: dW$. We want to know if the measures are absolutely continuous and what is the Radon-Nikodym derivative $dP/dQ$. When the measures are absolutely continuous with respect to each other, there is $g\in \F_t$ with $g>0$ such that for all $f\in \F_t$ we have
\begin{equation}\label{Radon}
E_{Q}[g f ]=E_{P}[f]\:.
\end{equation}
Our task is to find an explicit expression for $g$. Using that $g>0$, we can write $g = e^{L}$ and look for the SDE that $L$ satisfies. For $f=f(X)$ we obtain using the Ito formula under $Q$
\begin{equation}\label{Ito}
	\begin{split}
	d(g\:f)&=e^L \: f\: dL+e^L f_x dX + \frac{1}{2} e^L \: f \: dL\:dL+\frac{1}{2} e^L dX^t \:f_{xx} \: dX+e^L dL f_x \: dX\\
&=e^L \: f\: dL+\frac{1}{2} e^L \: f \: dL\:dL+\frac{1}{2} e^L \tr(\sigma^t \:f_{xx} \: \sigma) dt + e^L\:dL\:f_x\:\sigma\:dW+ \text{ Q-noise}.
	\end{split}
\end{equation}
Similarly, under $P$ we have
$$
df= f_x\:b\: dt+\frac{1}{2} \tr(\sigma^t \:f_{xx} \: \sigma) dt + \text{ P-noise}\:.
$$
Taking $f$ a constant function in \eqref{Radon} and \eqref{Ito} gives
\begin{equation}
	\begin{split}
	dL+\frac{1}{2}\:dL \: dL = \text{Q-noise}\:,
	\end{split}
\end{equation}
and taking $f$ linear we obtain
\begin{equation}
	\begin{split}
	dL\: \sigma\:dW+\text{Q-noise}=b \: dt +\text{P-noise}\:.
	\end{split}
\end{equation}
We write the SDE $dL=r\: dt + A \: dX$. Solving for $A$ and $r$ in the last two equations gives $A=b^t (\sigma \: \sigma^t)^{-1}$ and $r=-\frac{1}{2}\:b^t\:(\sigma \: \sigma^t )^{-1} b$\:.

\end{document}


