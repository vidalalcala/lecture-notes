\documentclass[10pt]{article}
%\documentclass[letter,10pt]{article}
\usepackage{fullpage}
% \usepackage{fullpage}
\usepackage{amsmath,amsfonts,amsthm,amssymb}
\usepackage{verbatim} % for comments
%\usepackage[notcite,notref]{showkeys} % shows all labels on the side
\usepackage{hyperref}

%The cool package for derivatives.
\usepackage{cool}
\Style{DSymb={\mathrm d},DShorten=true,IntegrateDifferentialDSymb=\mathrm{d}}


\numberwithin{equation}{section}

%%
%% Place here your \usepackage's. Some recommended packages are already included.
%%

% Graphics:
%\usepackage[final]{graphicx}
%\usepackage{graphicx} % use this line instead of the above to suppress graphics in draft copies
%\usepackage{graphpap} % \defines the \graphpaper command

% Indent first line of each section:
%\usepackage{indentfirst}

% Good AMS stuff:
%\usepackage{amsthm} % facilities for theorem-like environments
%\usepackage[tbtags]{amsmath} % a lot of good stuff!

% Fonts and symbols:
%\usepackage{amsfonts}
%\usepackage{amssymb}

% Formatting tools:
%\usepackage{relsize} % relative font size selection, provides commands \textsmalle, \textlarger
%\usepackage{xspace} % gentle spacing in macros, such as \newcommand{\acims}{\textsc{acim}s\xspace}

% Page formatting utility:
%\usepackage{geometry}

%%
%% Place here your \newcommand's and \renewcommand's. Some examples already included.
%%
\renewcommand{\le}{\leqslant}
\renewcommand{\ge}{\geqslant}
\renewcommand{\emptyset}{\ensuremath{\varnothing}}
\newcommand{\ds}{\displaystyle}
\newcommand{\R}{\ensuremath{\mathbb{R}}}
\newcommand{\Q}{\ensuremath{\mathbb{Q}}}
\newcommand{\Z}{\ensuremath{\mathbb{Z}}}
\newcommand{\N}{\ensuremath{\mathbb{N}}}
\newcommand{\T}{\ensuremath{\mathbb{T}}}
\newcommand{\eps}{\varepsilon}
\newcommand{\closure}[1]{\ensuremath{\overline{#1}}}

%MINE
\newcommand{\norm}[1]{\left\Vert#1\right\Vert}
\newcommand{\abs}[1]{\left\vert#1\right\vert}
\newcommand{\set}[1]{\left\{#1\right\}}
\newcommand{\epsi}{\varepsilon}
\newcommand{\To}{\longrightarrow}
\newcommand{\BX}{\mathbf{B}(X)}
\newcommand{\A}{\mathcal{A}}
\newcommand{\St}{\mathcal{S}}
\newcommand{\Par}{\mathcal{P}}
\newcommand{\La}{\mathcal{L}}
\newcommand{\F}{\mathcal{F}}
\newcommand{\Hld}{\mathcal{H}}
\newcommand{\Dst}{\mathcal{D}}

%LINEAR ALGEBRA
\newcommand{\tr}{\mathrm{tr}}

%PROBABILITY
\newcommand{\Var}{\mathrm{Var}\:}
\newcommand{\dis}{\mathrm{d}}
\newcommand{\iF}{\mathcal{F}}


%\newcommand{\acim}{\textsc{acim}\xspace}
%\newcommand{\acims}{\textsc{acim}s\xspace}

%%
%% Place here your \newtheorem's:
%%

%% Some examples commented out below. Create your own or use these...
%%%%%%%%%\swapnumbers % this makes the numbers appear before the statement name.
%\theoremstyle{plain}
%\newtheorem{thm}{Theorem}[chapter]
%\newtheorem{prop}[thm]{Proposition}
%\newtheorem{lemma}[thm]{Lemma}
%\newtheorem{cor}[thm]{Corollary}

%\theoremstyle{definition}
%\newtheorem{define}{Definition}[chapter]
%
%\theoremstyle{remark}
%\newtheorem*{rmk*}{Remark}
%\newtheorem*{rmks*}{Remarks}
%
%%% This defines the "proo" environment, which is the same as proof, but
%%% with "Proof:" instead of "Proof.". I prefer the former.
%\newenvironment{proo}{\begin{proof}[Proof:]}{\end{proof}}
\def\theequation{\thesection.\arabic{equation}}
\def\be{\begin{eqnarray}}
\def\ee{\end{eqnarray}}

\def\b*{\begin{eqnarray*}}
\def\e*{\end{eqnarray*}}

%
% Theorem/proposition/etc.
%
%%%%%%%%% numerotation %%%%%%%
\newtheorem{Theorem}{Theorem}[section]
\newtheorem{Definition}[Theorem]{Definition}
\newtheorem{Proposition}[Theorem]{Proposition}
\newtheorem{Property}[Theorem]{Property}
\newtheorem{Assumption}[Theorem]{Assumption}
\newtheorem{Lemma}[Theorem]{Lemma}
\newtheorem{Corollary}[Theorem]{Corollary}
\newtheorem{Remark}[Theorem]{Remark}
\newtheorem{Example}[Theorem]{Example}
% \newtheorem{Proof}{Proof}
% \newtheorem{Introduction}{Introduction}

\makeatletter \@addtoreset{equation}{section}
\def\theequation{\thesection.\arabic{equation}}
% \@addtoreset{Definition}{section}
% \def\theDefinition{\thesection.\arabic{Definition}}
% \@addtoreset{Theorem}{section}
% \def\theTheorem{\thesection.\arabic{Theorem}}
% \@addtoreset{Proposition}{section}
% \renewcommand{\theProposition}{\thesection.\arabic{Proposition}}
% \@addtoreset{Property}{section}
% \renewcommand{\theProperty}{\thesection.\arabic{Property}}
% \@addtoreset{Assumption}{section}
% \renewcommand{\theAssumption}{\thesection.\arabic{Assumption}}
% \@addtoreset{Corollary}{section}
% \renewcommand{\theCorollary}{\thesection.\arabic{Corollary}}
% \@addtoreset{Lemma}{section}
% \renewcommand{\theLemma}{\thesection.\arabic{Lemma}}
% \@addtoreset{Remark}{section}
% \renewcommand{\theRemark}{\thesection.\arabic{Remark}}
% \@addtoreset{Example}{section}
% \renewcommand{\theExample}{\thesection.\arabic{Example}}
%
% proof environment
%

%\newcommand{\ang}[1]{\left[#1\right]^\#}  % angular brackets for projection
%\newcommand{\brak}[1]{\left(#1\right)}    % round brackets
%\newcommand{\crl}[1]{\left\{#1\right\}}   % curly brackets
%\newcommand{\edg}[1]{\left[#1\right]}     % edgy brackets


\newcommand{\var}[1]{{\rm Var}\left(#1\right)}
\newcommand{\cov}[2]{{\rm Cov}\left(#1,#2\right)}
\newcommand{\No}[1]{\left\|#1\right\|}     % Norm

%%%%%%%%% tabulation %%%%%%%%%%%%%

%\addtolength{\oddsidemargin}{-0.1 \textwidth}
%\addtolength{\textwidth}{0.2 \textwidth}
%\addtolength{\topmargin}{-0.2 \textheight}
%\addtolength{\textheight}{0.5 \textheight}
%\addtolength{\parindent}{-0.02   \textwidth}
%\renewcommand{\baselinestretch}{1.1}

%%%% mathbb%%%%%%%%%%

\def \E{\mathbb{E}}
\def \F{\mathbb{F}}
\def \H{\mathbb{H}}
\def \L{\mathbb{L}}
\def \M{\mathbb{M}}
\def \N{\mathbb{N}}
\def \P{\mathbb{P}}
\def \Q{\mathbb{Q}}
\def \R{\mathbb{R}}
\def \S{\mathbb{S}}
\def \Z{\mathbb{Z}}

%%% cali %%%%

\def\Ac{\mathcal{A}}
\def\Bc{\mathcal{B}}
\def\Cc{\mathcal{C}}
\def\Dc{\mathcal{D}}
\def\Ec{\mathcal{E}}
\def\Fc{\mathcal{F}}
\def\Gc{\mathcal{G}}
\def\Hc{\mathcal{H}}
\def\Ic{\mathcal{I}}
\def\Jc{\mathcal{J}}
\def\Kc{\mathcal{K}}
\def\Lc{\mathcal{L}}
\def\Mc{\mathcal{M}}
\def\Nc{\mathcal{N}}
\def\Oc{\mathcal{O}}
\def\Pc{\mathcal{P}}
\def\Qc{\mathcal{Q}}
\def\Rc{\mathcal{R}}
\def\Sc{\mathcal{S}}
\def\Tc{\mathcal{T}}
\def\Uc{\mathcal{U}}
\def\Vc{\mathcal{V}}
\def\Wc{\mathcal{W}}
\def\Xc{\mathcal{X}}
\def\Yc{\mathcal{Y}}
\def\Zc{\mathcal{Z}}

%%% bar %%%%

\def\Ab{{\bar A}}
\def\Bb{{\bar B}}
\def\Cb{{\bar C}}
\def\Db{{\bar D}}
\def\Eb{{\bar E}}
\def\Fb{{\overline{F}}}
\def\Gb{{\bar G}}
\def\Hb{{\bar H}}
\def\Ib{{\bar I}}
\def\Jb{{\bar J}}
\def\Kb{{\bar K}}
\def\Lb{{\bar L}}
\def\Mb{{\bar M}}
\def\Nb{{\bar N}}
\def\Ob{{\bar O}}
\def\Pb{{\bar P}}
\def\Qb{{\bar Q}}
\def\Rb{{\bar R}}
\def\Sb{{\bar S}}
\def\Tb{{\bar T}}
\def\Ub{{\bar U}}
\def\Vb{{\bar V}}
\def\Wb{{\bar W}}
\def\Xb{{\bar X}}
\def\Yb{{\bar Y}}
\def\Zb{{\bar Z}}
\def\vo{\overline{v}}
%%% hat %%%%

\def\Ah{{\hat A}}
\def\Bh{{\hat B}}
\def\Ch{{\hat C}}
\def\Dh{{\hat D}}
\def\Eh{{\hat E}}
\def\Fh{{\hat F}}
\def\Gh{{\hat G}}
\def\Hh{{\hat H}}
\def\Ih{{\hat I}}
\def\Jh{{\hat J}}
\def\Kh{{\hat K}}
\def\Lh{{\hat L}}
\def\Mh{{\hat M}}
\def\Nh{{\hat N}}
\def\Oh{{\hat O}}
\def\Ph{{\hat P}}
\def\Qh{{\hat Q}}
\def\Rh{{\hat R}}
\def\Sh{{\hat S}}
\def\Th{{\hat T}}
\def\Uh{{\hat U}}
\def\Vh{{\hat V}}
\def\Wh{{\hat W}}
\def\Xh{{\hat X}}
\def\Yh{{\hat Y}}
\def\Zh{{\hat Z}}

%%% tilde %%%%

\def\At{{\tilde A}}
\def\Bt{{\tilde B}}
\def\Ct{{\tilde C}}
\def\Dt{{\tilde D}}
\def\Et{{\tilde E}}
\def\Ft{{\tilde F}}
\def\Gt{{\tilde G}}
\def\Ht{{\tilde H}}
\def\It{{\tilde I}}
\def\Jt{{\tilde J}}
\def\Kt{{\tilde K}}
\def\Lt{{\tilde L}}
\def\Mt{{\tilde M}}
\def\Nt{{\tilde N}}
\def\Ot{{\tilde O}}
\def\Pt{{\tilde P}}
\def\Qt{{\tilde Q}}
\def\Rt{{\tilde R}}
\def\St{{\tilde S}}
\def\Tt{{\tilde T}}
\def\Ut{{\tilde U}}
\def\Vt{{\tilde V}}
\def\Wt{{\tilde W}}
\def\Xt{{\tilde X}}
\def\Yt{{\tilde Y}}
\def\Zt{{\tilde Z}}


%%% gras %%%%

\def\HH{\mathbf{H}}
\def\FF{\mathbf{F}}
\def\LL{\mathbf{L}}
\def\TT{\mathbf{T}}


%%%% operateurs math %%%

\def\normep#1{\left|{#1}\right|_p}
%\def\E#1{E\left[{#1}\right]}
\def\Tr#1{{\rm Tr}\left[#1\right]}

\def \Sum{\displaystyle\sum}
\def \Prod{\displaystyle\prod}
\def \Int{\displaystyle\int}
\def \Frac{\displaystyle\frac}
\def \Inf{\displaystyle\inf}
\def \Sup{\displaystyle\sup}
\def \Lim{\displaystyle\lim}
\def \Liminf{\displaystyle\liminf}
\def \Limsup{\displaystyle\limsup}
\def \Max{\displaystyle\max}
\def \Min{\displaystyle\min}

\def\einf{{\rm ess \, inf}}
\def\esup{{\rm ess \, sup}}
\def\trace{{\rm Tr}}

%%%% texte dans formule %%%%%%%%%

\def\d#1{\mbox{\rm diag}\left[#1\right]}
\def\Pas{P-\mbox{a.s.}}
\def\And{\;\mbox{ and }\;}
\def\pourtout{\mbox{ for all }}

%%%%%% raccourci commandes %%%

\def\no{\noindent}
\def\x{\times}

\def\={\;=\;}
\def\.{\;.}

\def\vp{\varphi}
\def\eps{\varepsilon}

\def\reff#1{{\rm(\ref{#1})}}

%%%%%%% characteres

\def \i{1\!\mbox{\rm I}}
\def\1{\mathds{1}}

\def \ep{\hbox{ }\hfill{ ${\cal t}$~\hspace{-5.1mm}~${\cal u}$   } }

\def \proof{{\noindent \bf Proof. }}
\def \ep{\hbox{ }\hfill$\Box$}

\def \point{$\bullet$}

%%%%%%% propres articles

\def\pt{[\phi^N_t]}
\def\px{[\phi^N_x]}
\def\py{[\phi^N_y]}
\def\pz{[\phi^N_z]}

\def\fx{[f_x]}
\def\fy{[f_y]}
\def\fz{[f_z]}

\def\gx{[g_x]}

 \def\DW{\Delta W}

 \def\k#1{{k+ {#1}}}
 \def\n#1{{n+ {#1}}}
 \def\j#1{{j+ {#1}}}

 \def\Ek#1{E_k\left[{#1}\right]}
 \def\Ekh#1{\hat E_k\left[{#1}\right]}
 \def\Ekb#1{\bar E_k\left[{#1}\right]}
 \def\En#1{E_n\left[{#1}\right]}
 \def\Enh#1{\hat E_n\left[{#1}\right]}
 \def\Enb#1{\bar E_n\left[{#1}\right]}
 \def\normeL2#1{\left\|{#1}\right\|_{L^2}}
 %%%%%%%%%%%%%%%%%%%%%%%%%%%%%%%%%%%%%%%%%%%%%%%%%%%%%%%%%%%%%%%%%%%%%%%%%%%%%%%%%%%%%%%%%%%%
\newcommand {\lb} {\lambda}
\newcommand {\Chi} {{\bf \raise 1.5pt \hbox{$\delta$}}}
%
\newcommand {\partt} {\frac{\partial}{\partial t}}
\newcommand {\intOM} {\int_\Omega}
\newcommand{\lap}{\bigtriangleup}
\newcommand{\nb}{\nabla}
 %%%%%%%%%%%%%%%%%%%%%%%%%%%%%%%%%%%%%%%%%%%%%%%%%%%%%%%%%%%%%%%%%%%%%%%%%%%%%%%%%%%%%%%%%%%%

 \def\xhat{{\hat X}}
 
 %%PAPER definitions
 \newcommand{\s}{{\Sigma}}




%\newcommand{\footnotedva}[1]{\footnote{#1}} % comment this to hide the footnotes
\newcommand{\footnotedva}[1]{}

\DeclareMathOperator*{\argmin}{argmin\,}

\newtheorem{theorem}{Theorem}%[section]
\newtheorem{lemma}{Lemma}[section]
\newtheorem*{lemma*}{Lemma}
\newtheorem*{prop*}{Proposition}
\newtheorem*{cor*}{Corollary}

%\theoremstyle{remark}
\newtheorem*{remark*}{Remark}
\newtheorem{remark}[lemma]{Remark}

%%%%%%% hack for aligning enumerate with labels
\newenvironment{changemargin}[2]{%
  \begin{list}{}{%
    \setlength{\topsep}{0pt}%
    \setlength{\leftmargin}{#1}%
    \setlength{\rightmargin}{#2}%
    \setlength{\listparindent}{\parindent}%
    \setlength{\itemindent}{\parindent}%
    \setlength{\parsep}{\parskip}%
  }%
  \item[]}{\end{list}}
  
%opening
\begin{document}

\title{Stochastic Volatility}
\author{Jose V. Alcala-Burgos \\
ITESO\\
josealcala@iteso.mx}

\maketitle

\begin{abstract}

\end{abstract}


%%%%%%%%%%%%%%%%%%%%%%%%%%
%%%%%%%%%%%%%%%%%%%%%%%%%%

\section{Hedging in continuous time}
A \emph{hedging strategy} for an instrument with value $V$ is a portfolio $\Pi$ composed as follows.
\begin{itemize}
\item A short position on the instrument $V$.
\item $x_i$ shares of the instrument $S_i$.
\item $y$ shares of a risk free bond with value $A$.
\end{itemize}
The portfolio value satisfies
\begin{equation}
	\begin{split}
	\Pi &= \sum_{i} x_i S_i + y A - V\:,\qquad t\geq0,
	\end{split}
\end{equation}
and the initial condition
\begin{equation*}
	\begin{split}
	\Pi &= 0,\qquad t=0\:.	
	\end{split}
\end{equation*}
The objective is to find $x_i$ and $y$ such that the portfolio is \emph{self-financing}; that is,
\begin{equation}\label{self}
	\begin{split}
	\sum_{i} x_i dS_i+y dA-dV&=\sum_{i} dx_i S_i +dy A\:,\qquad t\geq 0
	\end{split}
\end{equation}
and the portfolio value does not change; that is,
\begin{equation}\label{constant}
	\begin{split}
	d\Pi = 0,\qquad t\geq 0\:.
	\end{split}
\end{equation}
We need models for $dS_i$ and $dV$ in order to decide if the previous equations hold. 

In the Black Scholes model we assume that
\begin{equation}\label{dS}
	\begin{split}
		dS &= \mu S dt +\sigma S dW\:,
	\end{split}
\end{equation}
and use the \emph{no arbitrage} hypothesis to derive the value of $V$ and $dV$. Namely, $V$ must be a function of $S$ and $t$; that is, $V = V(S,t)$. We will also \emph{assume} that $V$ has two continuous derivatives with respect to $S$ and one continuous derivative with respect to $t$. A portfolio $R$ with $x$ shares of $S$ and $y$ shares of $A$ is \emph{replicating} if and only if $\Pi=R-V$ is a hedging strategy. It follows that equations \eqref{self} and \eqref{constant} hold. We calculate, using Ito's formula,
\begin{equation}\label{dV}
	\begin{split}
	dV&= \biggl(V_t+\mu S V_S +\frac{1}{2} \sigma^2 S^2 V_{SS}\biggr)dt+\biggl(\sigma S V_S \biggr)dW\:,
	\end{split}
\end{equation}
and
\begin{equation}\label{dPi_1}
	\begin{split}
	d\Pi = x dS +y dA -dV + dx S+dy A\:.
	\end{split}
\end{equation}
Now substitute \eqref{self} and \eqref{dV} in \eqref{dPi_1}.
\begin{equation}\label{dPi_2}
	\begin{split}
	d\Pi &= 2 \biggl( x \mu S - \mu S V_S -V_t -\frac{1}{2} \sigma^2 S^2 V_{SS}+ r (V-xS) \biggr)dt\\
	&+ 2\biggl( x \sigma S - \sigma S V_S \biggr) dW\:.
	\end{split}
\end{equation}
In order to make both terms in equation \eqref{dPi_2} equal to zero we must pick $x = V_S$ and $V$ will be the solution of the Black-Scholes equation
\begin{equation}\label{BS}
	\begin{split}
	V_t + r S V_S + \frac{1}{2} \sigma^2 S^2 V_{SS}-rV&=0\:.
	\end{split}
\end{equation}
In the case of a European call option with strike price $K$ we obtain the Black-Scholes formula,
\begin{equation*}
	\begin{split}
	V = S N(d_1)-e^{-r(T-t)}K N(d_2)\:,
	\end{split}
\end{equation*}
with
\begin{equation*}
	\begin{split}
	d_1 &= \frac{\ln(S/K)+\bigl(r+\frac{1}{2} \sigma^2\bigr)(T-t)}{\sigma \sqrt{T-t}}\\
	d_2 &= d_1 - \sigma \sqrt{T-t}\:.
	\end{split}
\end{equation*}
These equations can be rewritten in terms of the \emph{forward price} $F(t,T)$ as
\begin{equation}
	\begin{split}
	V &= e^{-r(T-t)}(e^{r(T-t)}S N(d_1) - K N(d_2))\\
	   &= e^{-r(T-t)}(F(t,T) N(d_1) - K N(d_2))\:,
	\end{split}
\end{equation}
and
\begin{equation}
	\begin{split}
	d_1 &= \frac{\ln(e^{r(T-t)}S/K)+\frac{1}{2} \sigma^2(T-t)}{\sigma \sqrt{T-t}}\\
	d_1 &= \frac{\ln(F(t,T)/K)+\frac{1}{2} \sigma^2(T-t)}{\sigma \sqrt{T-t}}\\
	d_2 &= d_1 - \sigma \sqrt{T-t}\:.
	\end{split}
\end{equation}
The formulae in terms of the forward price is known as \emph{Black's formula}.

\section{The SABR stochastic volatility model ($\beta=1$)}
A stochastic volatility models has an extra source of randomness that impacts the stock price volatility. The SABR model for the stock price is defined by equation \eqref{dS} and the additional equation for the annualized volatility
\begin{equation}\label{dsigma}
	\begin{split}
	d\sigma = \gamma dt + \nu \sigma dZ \:,
	\end{split}
\end{equation}
where $Z$ is a Brownian motion such that $dWdZ=\rho dt$. This approach has two crucial properties:
\begin{enumerate}
\item The option price $V$ is now a function of $S,\sigma$ and $t$.
\item We need more than one risky asset in the hedging strategy. The usual choice is a \emph{liquid} option on the same underlying with value $U=U(S,\sigma,t)$ and the equation for the hedging portfolio, after we add $z$ units of $U$, is
\begin{equation*}
	\begin{split}
	\Pi = x S + z U + y A - V\:.
	\end{split}
\end{equation*}
\end{enumerate}

We can write down an equation for the dynamics of $\Pi$ using Ito's lemma:\begin{equation}\label{dV_vol}
	\begin{split}
	dV&= \biggl(V_t+\mu S V_S +\frac{1}{2} \sigma^2 S^2 V_{SS} + \frac{1}{2} \nu^2 \sigma^2 V_{\sigma\sigma}+\rho \sigma^2 \nu S V_{S\sigma} \biggr)dt + \sigma S V_S dW +(\gamma + \nu \sigma) V_{\sigma} dZ\:,
	\end{split}
\end{equation}
\begin{equation}\label{dU_vol}
	\begin{split}
	dU&= \biggl(U_t+\mu S U_S +\frac{1}{2} \sigma^2 S^2 U_{SS} + \frac{1}{2} \nu^2 \sigma^2 U_{\sigma\sigma} + \rho \sigma^2 \nu S U_{S\sigma}\biggr)dt + \sigma S U_S dW +(\gamma + \nu \sigma) U_{\sigma} dZ\:,
	\end{split}
\end{equation}
and
\begin{equation}\label{dPi_1_vol}
	\begin{split}
	d\Pi = x dS + z dU +y dA -dV + dx S  + dz U + dy A\:.
	\end{split}
\end{equation}
Now substitute \eqref{self}, \eqref{dV_vol}, and \eqref{dU_vol} in \eqref{dPi_1_vol}.
\begin{equation}\label{dPi_2_vol}
	\begin{split}
	d\Pi &= 2 \biggl( x \mu S + z \biggl[U_t +  \mu S U_{S} + \frac{1}{2} \sigma^2 S^2 U_{SS} + \frac{1}{2} \nu^2 \sigma^2 U_{\sigma\sigma} + \rho \sigma^2 \nu S U_{S\sigma}\biggr] \\
	&  - \biggl[ V_t  + \mu S V_S + \frac{1}{2} \sigma^2 S^2 V_{SS} + \frac{1}{2} \nu^2 \sigma^2 V_{\sigma\sigma} + \rho \sigma^2 \nu S V_{S\sigma} \biggr] \\
	& + r (V-xS-zU) \biggr)dt\\
	& + 2\biggl( x \sigma S + z \sigma S U_{S}  - \sigma S V_S \biggr) dW  + 2 (\gamma+\nu \sigma) \biggl( z U_{\sigma} -  V_{\sigma} \biggr) dZ\:.
	\end{split}
\end{equation}
The only way the third term in equation \eqref{dPi_2_vol} vanishes is with the choice
\begin{equation}
	\begin{split}
	z =& \frac{V_{\sigma}}{U_{\sigma}}\:;
	\end{split}
\end{equation}
thus, we can not hedge the option $V$ with only the underlying $S$. In order to make the last two terms in equation \eqref{dPi_2} equal to zero we must choose
\begin{equation*}
	\begin{split}
		x &= V_S - z U_S \:.
	\end{split}
\end{equation*}
With this choice of $x$ and $z$ we obtain
\begin{equation*}
	\begin{split}
	d\Pi &= 2 \biggl(\frac{V_{\sigma}}{U_{\sigma}}\biggl[U_t + r S U_S + \frac{1}{2} \sigma^2 S^2 U_{SS}+ \frac{1}{2} \nu^2 \sigma^2 U_{\sigma\sigma}  + \rho \sigma^2 \nu S U_{S\sigma} - r U\biggr] \\
	& -\biggl[ V_t + r S V_S + \frac{1}{2} \sigma^2 S^2 V_{SS} + \frac{1}{2} \nu^2 \sigma^2 V_{\sigma\sigma} + \rho \sigma^2 \nu S V_{S\sigma} - r V \biggr] \biggr)dt\:.
	\end{split}
\end{equation*}
The RHS of the last equation vanishes iff there is a function $\phi(S,\sigma,t)$ such that
\begin{equation*}
	\begin{split}
	U_t + r S U_S + \frac{1}{2} \sigma^2 S^2 U_{SS}+ \frac{1}{2} \nu^2 \sigma^2 U_{\sigma\sigma}  + \rho \sigma^2 \nu S U_{S\sigma} - r U&=\phi U_{\sigma}\\
	V_t + r S V_S + \frac{1}{2} \sigma^2 S^2 V_{SS}+ \frac{1}{2} \nu^2 \sigma^2 V_{\sigma\sigma}  + \rho \sigma^2 \nu S V_{S\sigma} - r V &=\phi V_{\sigma}\:.
	\end{split}
\end{equation*}
The function $\phi$ is written in terms of the \emph{market price of volatility risk} $\lambda(S,\sigma,t)$ as
\begin{equation*}
	\begin{split}
	\phi =- (\gamma-\lambda \nu \sigma)\:.
	\end{split}
\end{equation*}
and the final PDE for $V$ (and $U$) becomes
\begin{equation}\label{PDE_vol_risk}
	\begin{split}
	V_t + r S V_S +(\gamma- \lambda \nu \sigma) V_{\sigma}+ \frac{1}{2} \sigma^2 S^2 V_{SS}+ \frac{1}{2} \nu^2 \sigma^2 V_{\sigma\sigma}  + \rho \sigma^2 \nu S V_{S\sigma} - r V&=0\\
	\end{split}
\end{equation}
This is the same PDE satisfied by
\begin{equation*}
	\begin{split}
	V(S,\sigma,t) = E_{S,\sigma,t} [e^{-r(T-t)} f(S_T,\sigma_T,T)] \:,
	\end{split}
\end{equation*}
assuming the \emph{risk neutral dynamics} for the variables $S$ and $\sigma$
\begin{equation}\label{RN}
	\begin{split}
		dS &= r S dt +\sigma S dW^{*}\\
		d\sigma &= (\gamma - \lambda \nu \sigma)  dt + \nu\sigma dZ^{*}\:,
	\end{split}
\end{equation}
and the condition $\rho\leq 0$. The case $\rho$ is analyzed in section \eqref{rho}. \textbf{Each choice of the function $\lambda(S,\sigma,t)$ gives a rule to calculate the values of all options}. Moreover, assuming all prices are calculated with the same $\lambda$, it is possible to hedge an option using the underlying and another option on the same underlying.

We will assume
\begin{itemize}
\item $\gamma=\overline{\gamma}\sigma$.
\item $\lambda$ is a constant.
\end{itemize}
Under this model, a \emph{forward contract} has present value equal to
\begin{equation*}
	\begin{split}
	e^{-r(T-t)}E_{*}[S_T- F(t,T) | \iF_t] &= e^{-r(T-t)}[e^{r(T-t)}S_t-F(t,T)]\:,
	\end{split}
\end{equation*}
and therefore $F(t,T)=e^{r(T-t)}S_t$. In the same way, an imaginary forward contract with payoff $\sigma_T - G(t,T)$ has present value
\begin{equation*}
	\begin{split}
	e^{-r(T-t)}E_{*}[\sigma_T- G(t,T) | \iF_t] &=e^{-r(T-t)} [E_{*}[\sigma_T | \iF_t] - G(t,T)]\:,
	\end{split}
\end{equation*}
and therefore $G(t,T)=e^{(\overline{\gamma}-\lambda \nu)(T-t)}\sigma_t $.

Finally, we can write down equations in terms in terms of the \emph{forward variables}
\begin{equation*}
	\begin{split}
	 \widehat{F_t} &= e^{r(T-t)} S_t \\
	 \widehat{\alpha}_t &= e^{(\overline{\gamma}-\lambda \nu)(T-t)}\sigma_t \:. 
	\end{split}
\end{equation*}
The result is
\begin{equation}\label{SABR}
	\begin{split}
		d\widehat{F} &= e^{-(\overline{\gamma}-\lambda \nu)(T-t)}\widehat{\alpha} \widehat{F} dW^{*}\\
		d \widehat{\alpha }&= \nu \widehat{\alpha} dZ^{*}\:.
	\end{split}
\end{equation}
Hagan considers the case $\overline{\gamma}-\lambda \nu=0$ and the PDE in the forward variables is
\begin{equation}\label{Hagan_V}
	\begin{split}
	V_t + \frac{1}{2} a^2 F^2 V_{FF}+ \frac{1}{2} \nu^2 a^2 V_{aa}  + \rho a^2 \nu F V_{Fa} - r V&=0\\
	\end{split}
\end{equation}
Notice that at maturity the forward variables match the the original variables; that is, $F_T=S_T$ and $\widehat{\alpha}_T = \sigma_{T}$. Therefore the call value satisfies
\begin{equation*}
	\begin{split}
		C_{T} &= (S_T - K)_{+}= (\widehat{F}_T-K)_{+}
	\end{split}
\end{equation*}
If the underlying is fairly liquid we will have quotes for $\widehat{F}_{t}$ using the current \emph{futures price} of the underlying. The \emph{forward volatility} $\widehat{\alpha}$ is harder to quote and the standard practice is to add its current value as the parameter $\alpha$ of the model.

\section{Hagan's SABR approximation}
European option market prices are quoted with the implied volatility $\s$. The relationship between $V$ and $\s$ is summarized in the equations
\begin{equation}
	\begin{split}
	V &= e^{-r(T-t)}(F N(d_1) - K N(d_2))\:,
	\end{split}
\end{equation}
and
\begin{equation}
	\begin{split}
	d_1 &= \frac{\ln(F/K)+\frac{1}{2} \s^2(T-t)}{\s \sqrt{T-t}}\\
	d_2 &= d_1 - \s \sqrt{T-t}\:.
	\end{split}
\end{equation}
Since $V=V(F,a,t)$ we have $\s = \s(F,a,t)$. Moreover, we can get a PDE for $\s$ using the PDE \eqref{Hagan_V}. The result is
\begin{equation*}
	\begin{split}
	\s_t &+\frac{1}{2}a^2 F^2\s_{FF}+\frac{1}{2}\nu^2 a^2 \s_{aa}+\rho a^2 \nu F \s_{F a}\\
	& + \frac{d_2}{\s}\biggl\{ \frac{1}{2}a^2 F \bigl[ F d_1 (\s_{F})^{2}-2\frac{\s_{F}}{\sqrt{T-t}} \bigr] \\
	&+ \frac{1}{2} v^2 a^2 d_1 (\s_{a})^2\\
	&+ \rho a^2 \nu \bigl[ F d_1 \s_{F}\s_{a} - \frac{\s_{a}}{\sqrt{T-t}} \bigr] \biggr\} \:.
	\end{split}
\end{equation*}
This is a nonlinear PDE and we will use it soon to find an approximation of $\Sigma$, but we need to find the boundary condition at the final time for $\Sigma$.

\section{Implied volatility close to maturity.}
We start with the function $W=E_{*}[(F-K)_{+}]$ which satisfies the PDE
\begin{equation}\label{Hagan_W}
	\begin{split}
	W_t + \frac{1}{2} a^2 F^2 W_{FF}+ \frac{1}{2} \nu^2 a^2 W_{aa}  + \rho a^2 \nu F W_{Fa} &=0\:,
	\end{split}
\end{equation}
with the boundary condition
\begin{equation}\label{bndry_W}
	\begin{split}
	W(F,a,T)=  (F-K)_{+}	
	\end{split}
\end{equation}
Taking a time derivative on both sides of \eqref{Hagan_W}, we can see that $W_t$ also satisfies the PDE \eqref{Hagan_W}. Now substitute \eqref{bndry_W} in \eqref{Hagan_W}. The result is the boundary condition $W_t(F,K,T)+\frac{1}{2}a^2 K^2 \delta_{(F-K)}=0$. We can \emph{approximate} $W_t$ close to maturity as a multiple of a Gaussian function. 


\section{Asymptotic expansion over $\nu$ }

The PDE we use to calculate option prices is
\begin{equation}\label{SABR_PDE}
	\begin{split}
	W_t + \frac{1}{2} a^2 F^2 W_{FF}+ \frac{1}{2} \nu^2 a^2 W_{aa}  + \rho a^2 \nu F W_{Fa}&=0\\
	W(F,a,T)&=(F-K)_{+}\:,
	\end{split}
\end{equation}
where $W=E_{*}[(\widehat{F}_{T}-K)_{+}]$. We will try to approximate the solution of the previous equation with an asymptotic expansion
\begin{equation}\label{expansion}
	\begin{split}
	W(F,a,t;\nu) = W^{0}(F,a,t) + W^{1}(F,a,t) \nu + W^{2}(F,a,t) \nu^2 + O(\nu^3)
	\end{split}
\end{equation}
Substitute \eqref{expansion} in \eqref{SABR_PDE} and collect $O(1)$ terms. The result is
\begin{equation}\label{order_0}
	\begin{split}
	W_t^{0} + \frac{1}{2}  a^2 F^2 W_{FF}^{0}&=0\\
	W^{0}&=(F-K)_{+}\:;\qquad t=T\:.
	\end{split}
\end{equation}
There are no derivatives w.r.t $a$ in the previous equation, therefore the solution is given by Black's formula with $a$ playing the role of the volatility; this is,
\begin{equation*}
	\begin{split}
	W^{0}&= FN(d_1)-KN(d_2)\\
	d_{1}&=\frac{\ln\bigl( \frac{F}{K}\bigr)+\frac{1}{2} a^2 (T-t)}{a\sqrt{T-t}}\:.
	\end{split}
\end{equation*}
For convenience we will denote $\tau = T -t$. Now we collect $O(\nu)$ terms in equation \eqref{SABR_PDE}. The result is
\begin{equation}
	\begin{split}
	W_t^{1} + \frac{1}{2}  a^2 F^2 W_{FF}^{1}+\rho a^2 F W_{Fa}^{0}&=0\\
	W^{1}&=0\:,\qquad t=T\:.
	\end{split}
\end{equation}
We calculate $W_{Fa}^{0}$ using the financial greeks. After rearranging terms, the previous equation becomes
\begin{equation}\label{order_1}
	\begin{split}
	W_t^{1} + \frac{1}{2}  a^2 F^2 W_{FF}^{1} &=\rho a  F d_2 \frac{e^{-d_1^2 / 2}}{\sqrt{2\pi}}\\
	W^{1}&=0\:,\qquad t=T\:.
	\end{split}
\end{equation}
Finally we define $y=d_2 \sqrt{\tau} = \frac{ \ln\bigl( \frac{F}{K}\bigr) - \frac{1}{2} a^2 (T-t)}{a}$ and the PDE in the new variables reads
\begin{equation}\label{order_1}
	\begin{split}
	W_t^{1} + \frac{1}{2}  W_{yy}^{1}&= -\rho a K \tau \frac{d^2}{dy^2}N(y/\sqrt{\tau}) \\
	W^{1}&=0\:,\qquad t=T\:.
	\end{split}
\end{equation}
Now we can use the heat kernel to calculate the solution of the previous \emph{non-homogeneous} heat equation. The key relationships we will use in the next calculations are
\begin{equation}\label{heat_kernel}
	\begin{split}
	\int_{-\infty}^{\infty}\frac{e^{-\frac{(y-x)^2}{2 t}}}{\sqrt{2\pi t}} \frac{d^k}{dy^k} N(y/\sqrt{s})dy&=\frac{d^k}{dx^k} N(x/\sqrt{s+t}) \:, \qquad k>1\\
	\int_{-\infty}^{\infty}\frac{e^{-\frac{(y-x)^2}{2 t}}}{\sqrt{2\pi t}} \:y\:\frac{d^k}{dy^k} N(y/\sqrt{s})dy&=x \frac{d^k}{dx^k} N(x/\sqrt{s+t}) -\frac{t}{s+t} \frac{d^{k-1}}{dx^{k-1}}\biggl[ x \frac{d}{dx}N \bigl( x/\sqrt{s+t}\bigr)\biggr]\:, \qquad k>1\\
	\int_{-\infty}^{\infty}\frac{e^{-\frac{(y-x)^2}{2 t}}}{\sqrt{2\pi t}} \:y\:\frac{d^k}{dy^k} N(y/\sqrt{s})dy&=\frac{s}{s+t}\:x \frac{d^k}{dx^k} N(x/\sqrt{s+t}) -\frac{t}{s+t}(k-1) \frac{d^{k-1}}{dx^{k-1}}N \bigl( x/\sqrt{s+t}\bigr)\:, \qquad k>1
	\end{split}
\end{equation}
Duhamel's principle gives
\begin{equation*}
	\begin{split}
	W^{1}&=\int_{t}^{T}\int_{-\infty}^{\infty} \rho a K (T-s) \frac{d^2}{dz^2}N(z/\sqrt{T-s}) \frac{e^{-\frac{(z-y)^2}{2 (s-t)}}}{\sqrt{2\pi (s-t)}} dz ds\\
	&=\int_{t}^{T} \rho a K (T-s) \frac{d^2}{dy^2}N(y/\sqrt{T-t})ds\\
	&=\rho a K \frac{d^2}{dy^2}N(y/\sqrt{T-t})\int_{t}^{T}  (T-s) ds\\
	&=\rho a K \frac{d^2}{dy^2}N(y/\sqrt{T-t})\frac{1}{2}(T-t)^2\:.
	\end{split}
\end{equation*}
In other words,
\begin{equation*}
	\begin{split}
	W^{1}&=\frac{1}{2}\rho a K \tau^2 \frac{d^2}{dy^2}N(y/\sqrt{\tau})\:.
	\end{split}
\end{equation*}
Going back to the original variables, we have found the first correction; this is,
\begin{equation*}
	\begin{split}
	W&=FN(d_1)-KN(d_2)+\nu \frac{\rho a K \tau }{2} d_2 \frac{e^{-\frac{d_2^2}{2}} }{\sqrt{2\pi}} +O(\nu^2)\:.
	\end{split}
\end{equation*}
We continue the approximation by collecting $O(\nu^2)$ terms in equation \eqref{SABR_PDE}. The result is
\begin{equation*}
	\begin{split}
	W^{2}_{t} +\frac{1}{2} a^2 F^2 W^{2}_{FF}+\frac{1}{2} a^2 W^{0}_{aa}+\rho a^2 F W^{1}_{Fa}&=0\:.
	\end{split}
\end{equation*}
Using the explicit formulae for $W^{0}$ and $W^{1}$ and the finance greeks we calculate
\begin{equation*}
	\begin{split}
	  W_{a}^{0} &= K\tau \frac{d}{dy} N(y/\sqrt{\tau})\\
	  W_{aa}^{0} &= K\tau\biggl( \tau - \frac{y}{a}\biggr) \frac{d^2}{d y^2} N(y/\sqrt{\tau})\\
	  W_{F}^{1} &= -\frac{1}{2}\rho \frac{K}{F} \tau^2 \frac{d^3}{dy^3}N(y/\sqrt{\tau})\\
	    W_{Fa}^{1} &= -\frac{1}{2}\rho \frac{K}{F} \tau^2 \biggl(\tau-\frac{y}{a}\biggr) \frac{d^4}{dy^4}N(y/\sqrt{\tau})
	\end{split}
\end{equation*}
In terms of the variable $y$ we obtain
\begin{equation*}
	\begin{split}
	W^{2}_{t} +\frac{1}{2} W^{2}_{yy} &= -\frac{1}{2}aK\tau ( a\tau - y ) \frac{d^2}{d y^2} N(y/\sqrt{\tau}) +\frac{1}{2}\rho^2 a K \tau^2 (a\tau-y ) \frac{d^4}{dy^4}N(y/\sqrt{\tau})\:.
	\end{split}
\end{equation*}
Duhamel's principle and equation \eqref{heat_kernel} give
%\begin{equation*}
%	\begin{split}
%	W^{2}(x,a,t_0) &= \int_{t_0}^{T}\biggl[  \frac{1}{2}aK\tau ( a\tau - x ) \frac{d^2}{d x^2} N(x/\sqrt{T-t_0}) -\frac{1}{2}\rho^2 a K \tau^2 (a\tau-x ) \frac{d^4}{dx^4}N(x/\sqrt{T-t_0})\\
%	&+ \frac{1}{2}aK\frac{\tau^2}{T-t_0} \frac{d}{dx}\biggl( x \frac{d}{dx}N\biggl( \frac{x}{\sqrt{T-t_0}}\biggr)\biggr)\\ 
%	&- \frac{1}{2}\rho^2 aK  \frac{\tau^3}{T-t_0} \frac{d^{3}}{dx^3}\biggl( x \frac{d}{dx}N\biggl( \frac{x}{\sqrt{T-t_0}}\biggr)\biggr) \biggr]\:dt\\
%	&= \int_{0}^{T-t_0}\biggl[  \frac{1}{2}aK\tau ( a\tau - x ) \frac{d^2}{d x^2} N(x/\sqrt{T-t_0}) -\frac{1}{2}\rho^2 a K \tau^2 (a\tau-x ) \frac{d^4}{dx^4}N(x/\sqrt{T-t_0})\\
%	&+ \frac{1}{2}aK \frac{\tau^2}{T-t_0} \frac{d}{dx}\biggl( x \frac{d}{dx}N\biggl( \frac{x}{\sqrt{T-t_0}}\biggr)\biggr)\\ 
%	&- \frac{1}{2}\rho^2 aK \frac{\tau^3}{T-t_0} \frac{d^{3}}{dx^3}\biggl( x \frac{d}{dx}N\biggl( \frac{x}{\sqrt{T-t_0}}\biggr)\biggr) \biggr]\:d\tau\\
%	&= \biggl[  \frac{1}{2}aK\tau^2 ( a\tau/3 - x/2 ) \frac{d^2}{d x^2} N(x/\sqrt{T-t_0}) -\frac{1}{2}\rho^2 a K \tau^3 (a\tau/4-x/3 ) \frac{d^4}{dx^4}N(x/\sqrt{T-t_0})\\
%	&+ \frac{1}{6}aK \frac{\tau^3}{T-t_0} \frac{d}{dx}\biggl( x \frac{d}{dx}N\biggl( \frac{x}{\sqrt{T-t_0}}\biggr)\biggr)\\ 
%	&- \frac{1}{8}\rho^2 aK \frac{\tau^4}{ T-t_0} \frac{d^{3}}{dx^3}\biggl( x \frac{d}{dx}N\biggl( \frac{x}{\sqrt{T-t_0}}\biggr)\biggr) \biggr]_{\tau = 0}^{\tau = T-t_0}\:.\\
%	\end{split}
%\end{equation*}
%In other words
\begin{equation*}
	\begin{split}
	W^{2}&=  \frac{1}{6} a K \tau^2 (a\tau - y) \frac{d^2}{d y^2} N(y/\sqrt{\tau})
	- \frac{1}{8} \rho^2 a K \tau^3 ( a\tau - y )  \frac{d^4}{d y^4} N(y/\sqrt{\tau})\\
	&+\frac{1}{12} a K \tau^2 \frac{d}{d y} N(y/\sqrt{\tau})
	-\frac{1}{8} \rho^2 \tau^3 a K \frac{d^3}{d y^3} N(y/\sqrt{\tau})
	\end{split}
\end{equation*}

\bibliographystyle{plain}	% (uses file "plain.bst")
\bibliography{calibration}
\end{document}  